\input pdfToolbox
\input syntax-hi
\input preamble

\setlayout{horizontal margin=2cm, vertical margin=2cm}

{\setbox0=\hbox{\setfontandscale{bf}{25pt}Linear Reduction}
\centerline{
    \vbox{
        \copy0
        \smallskip
        %\hbox to\wd0{\hfil Ari Feiglin}
    }
}}

\bigskip
\hbox to\hsize{\hfil\vbox{\hsize=.7\hsize
\leftskip=0pt plus 1fil \rightskip=\leftskip \parfillskip=\z@
\hrule
\kern5pt

In this paper I will define the concept of linear reduction in the context of syntax parsing.
We will progress through more and more complicated examples, beginning from the programming of a simple calculator until we ultimately have created an extensible programming language.

\kern5pt
\hrule
}\hfil}

\section*{Table of Contents}

\tableofcontents

\vfill\break

\setcounter{section}{-1}
\section{Notation}

\noindent ${\bb N}$ denotes the set of natural numbers, including $0$.

\smallskip
\noindent $\overline{\bb N}$ is defined to be ${\bb N}\cup\set\infty$.

\smallskip
\noindent $f\colon A\plongto B$ means that $f$ is a partial function from $A$ to $B$.

\vfill\break

\section{Theoretical Background}

\subsection{Stateless Reduction}

The idea of linear reduction is simple: given a string $\xi$ the first character looks if it can bind with the second character to produce a new character, and the process repeats itself.
There is of course, nuance.
This nuance hides in the statement ``if it can bind'': we must define the rules for binding.

Let us define an {\it reducer} to be a tuple $(\Sigma,\beta,\pi)$ where $\Sigma$ is an alphabet; $\beta\colon\overline\Sigma\times\overline\Sigma\plongto\overline\Sigma$ is a partial function called the
{\it reduction function} where $\overline\Sigma=\Sigma\times\overline{\bb N}$; and $\pi$ is the {\it initial priority function}.
A {\it program} over an reducer is a string over $\overline\Sigma$.
We write a program like $\sigma^1_{i_1}\cdots\sigma^n_{i_n}$ instead of as pairs $(\sigma^1,i_1)\dots(\sigma^n,i_n)$.
In the character $\sigma_i$, we call $i$ the {\it priority} of $\sigma$.

Then the rules of reduction are as follows, meaning we define $\beta(\xi)$ for a program:
We do so in cases:
\benum
    \item If $\xi=\sigma_i$ then $\beta(\xi)=\sigma_0$.
    \item If $\xi=\sigma^1_i\sigma^2_j\xi'$ where $i\geq j$ and $\beta(\sigma^1_i,\sigma^2_j)=\sigma^3_k$ is defined then $\beta(\xi)=\sigma^3_k\xi'$.
    \item Otherwise, for $\xi=\sigma^1_i\sigma^2_j\xi'$, $\beta(\xi)=\sigma^1_i\beta(\sigma^2_j\xi')$.
\eenum

A string $\xi$ such that $\beta(\xi)=\xi$ is called {\it irreducible}.
Notice that it is possible for a string of length more than $1$ to be irreducible: for example if $\beta(\sigma^1,\sigma^2)$ is not defined then $\sigma^1_i\sigma^2_j$ is irreducible.
$$ \beta(\sigma_1\tau_2)\xvarrightarrow{(3)}\sigma_1\beta(\tau_2)\xvarrightarrow{(1)}\sigma_1\tau_2 $$
But such strings are not desired, since in the end we'd like a string to give us a value.
So an irreducible string which is not a single character is called {\it ill-written}, and a string which is not ill-written is {\it well-written}.

Now the initial priority function is $\pi\colon\Sigma\plongto\overline{\bb N}$ which gives characters their initial priority.
We can then canonically extend this to a function $\pi\colon\Sigma^*\plongto(\Sigma\times\overline{\bb N})^*$ defined by $\pi(\sigma^1\cdots\sigma^n)=\sigma^1_{\pi(\sigma^1)}\cdots\sigma^n_{\pi(\sigma^n)}$.
Then a $\beta$-reduction of a string $\xi\in\Sigma^*$ is taken to mean a $\beta$-reduction of $\pi(\xi)$.

Notice that once again we require that $\pi$ only be a partial function.
This is since that we don't always need every character in $\Sigma$ to have an initial priority; some symbols are only given their priority through the $\beta$-reduction of another pair of symbols.
So we now provide a new definition of a {\it program}, which is a string $\xi=\sigma^1\cdots\sigma^n\in\Sigma^*$ such that $\pi(\sigma^i)$ exists for all $1\leq i\leq n$.
We can only of course discuss the reductions of programs, as $\pi(\xi)$ is only defined if $\xi$ is a program.

\medskip
{\bf Example:\/} let $\Sigma={\bb N}\cup\set{+,\cdot}\cup\set{(n+),(n\cdot)}[n\in{\bb N}]$.
$\beta$ as follows:

\medskip
\centerline{
    \vtop{\ialign{\hfil$#$\hfil\tabskip=.25cm&\hfil$#$\hfil\cr
        \sigma^1_i,\sigma^2_j & \beta(\sigma^1_i,\sigma^2_j)\cr\noalign{\kern3pt\hrule\kern3pt}
        n,+ & (n+)\cr
        n,\cdot & (n\cdot)\cr
        (n+),m & n+m\cr
        (n\cdot),m & n\cdot m\cr
        (n\cdot),(m+) & (n\cdot m,+)\cr
        (n+),(m+) & (n+m,+)\cr
        (n\cdot),(m\cdot) & (n\cdot m,\cdot)\cr
    }}
}
\medskip
\noindent Where $n,m$ range over all values in ${\bb N}$.
Here $\beta(\sigma_i,\sigma_j)$'s priority is $j$.
We define the initial priorities
$$ \pi(n) = \infty,\quad \pi(+) = 1,\quad \pi(\cdot) = 2 $$

Now let us look at the string $1+2\cdot3+4;$.
Here,
$$ \eqalign{
    1_\infty+_12_\infty\cdot_23_\infty+_14_\infty &\longto (1+)_12_\infty\cdot_23_\infty+_14_\infty\cr
    &\longto (1+)_1(2\cdot)_23_\infty+_14_\infty\cr
    &\longto(1+)_1(2\cdot)_2(3+)_14_\infty\cr
    &\longto(1+)_1(6+)_14_\infty\cr
    &\longto(7+)_14_\infty\cr
    &\longto(7+)_14_0\cr
    &\longto(11)_0
} $$
So the rules for $\beta$ we supplied seem to be sufficient for computing arithmetic expressions following the order of operations.
\qedd

{\bf Example:\/} We can also expand our language to include parentheses.
So our alphabet becomes $\Sigma={\bb N}\cup\set{+,\cdot,(,)}\cup\set{\uline{ n+},\uline{ n\cdot},\uline{ n)}}[n\in{\bb N}]$.
We distinguish between parentheses and bold parentheses for readability.
We extend $\beta$ as follows:

\medskip
\centerline{
    \vtop{\ialign{\hfil$#$\hfil\tabskip=.25cm&\hfil$#$\hfil\cr
        \sigma^1_i,\sigma^2_j & \beta(\sigma^1_i,\sigma^2_j)\cr\noalign{\kern3pt\hrule\kern3pt}
        n,+ & \uline{ n+}_j\cr
        n,\cdot & \uline{ n\cdot}_j\cr
        \uline{ n+},m & (n+m)_j\cr
        \uline{ n\cdot},m & (n\cdot m)_j\cr
        \uline{ n\cdot},\uline{ m+} & \uline{ n\cdot m,+}_j\cr
        \uline{ n+},\uline{ m+} & \uline{ n+m,+}_j\cr
        \uline{ n\cdot},\uline{ m\cdot} & \uline{ n\cdot m,\cdot}_j\cr
        n,) & \uline{ n)}_j\cr
        \uline{ n+},\uline{ m)} & \uline{ n+m)}_j\cr
        \uline{ n\cdot},\uline{ m)} & \uline{ n\cdot m)}_j\cr
        (,\uline{ n)} & n_i\cr
    }}
}
\medskip

\noindent $(n+m)_j$ means $n+m$ with a priority of $j$, not $\uline{ n+m}_j$.
And we define the initial priorities
$$ \pi(n) = \infty,\quad \pi(+) = 1,\quad \pi(\cdot) = 2,\quad \pi({(}) = \infty,\quad \pi({)}) = 0 $$
So for example reducing $2\cdot((1+2)\cdot2)+1$,
$$ \eqalign{
    2_\infty*_2{(_\infty}{(_\infty}1_\infty+_12_\infty{)_0}*_22_\infty{)_0}+_11_\infty &\longto \uline{2*}_2{(_\infty}{(_\infty}1_\infty+_12_\infty{)_0}*_22_\infty{)_0}+_11_\infty\cr
    &\longto \uline{2*}_2{(_\infty}{(_\infty}\uline{1+}_12_\infty{)_0}*_22_\infty{)_0}+_11_\infty\cr
    &\longto \uline{2*}_2{(_\infty}{(_\infty}\uline{1+}_1\uline{2)}_0*_22_\infty{)_0}+_11_\infty\cr
    &\longto \uline{2*}_2{(_\infty}{(_\infty}\uline{3)}_0*_22_\infty{)_0}+_11_\infty\cr
    &\longto \uline{2*}_2{(_\infty}3_\infty*_22_\infty{)_0}+_11_\infty\cr
    &\longto \uline{2*}_2{(_\infty}\uline{3*}_22_\infty{)_0}+_11_\infty\cr
    &\longto \uline{2*}_2{(_\infty}\uline{3*}_2\uline{2)}_0+_11_\infty\cr
    &\longto \uline{2*}_2{(_\infty}\uline{6)}_0+_11_\infty\cr
    &\longto \uline{2*}_26_\infty+_11_\infty\cr
    &\longto \uline{2*}_2\uline{6+}_11_\infty\cr
    &\longto \uline{12+}_11_\infty\cr
    &\longto \uline{12+}_11_0\cr
    &\longto 13_0\cr
} $$
\qedd

\subsection{Stateful Reduction}

Suppose we'd like to reduce a program with variables in it.
Then we cannot just use the previous definitions, as the actions of $\sigma$ (which is to be understood as the function $\beta(\sigma,\bullet)$) are determined before any reduction occurs.
We need a way to store the value of variables, a state.

This leads us to the following definition: let $\Sigma_P$ and $\Sigma_A$ be two disjoint sets of symbols: $\Sigma_P$ the set of {\it printable symbols} and $\Sigma_A$ the set of {\it abstract symbols}.
$\Sigma_P$ will generally be a set consisting of the string representations of abstract symbols, be it operators like $+$ and $\cdot$ or variable names.
$\Sigma_A$ are the actual objects which can ``execute something''.
Let us further define $\Sigma=\Sigma_P\cup\Sigma_A$.

Now a state is a mapping from printable symbols to strings.
So for example, if $x$ is a printable symbol a line like $\lett x=1$ should change the state so that $x$ maps to the abstract symbol representing $1$.
So we can view a state as a map from $\Sigma_P\longto\Sigma^*$, where if the map of a symbol is $\epsilon$ (the empty word), this represents it not having a value.
But in order to implement locality, we must be able to revert to a previous state.
Thus a state will actually be a (non-empty) sequence of maps $\Sigma_P\longto\Sigma^*$.
Maps $\Sigma_P\longto\Sigma^*$ are called {\it point states}.
A {\it substitution} is a partial map $\nu\colon\Sigma_P\plongto\Sigma^*$ which represents changing the values in its domain to their new values in the substitution.
For a point-state $\hat s$ define $\hat s\nu$ by $\hat s\nu(\sigma)=\nu(\sigma)$ if $\sigma$ is in $\nu$'s domain and $\hat s(\sigma)$ otherwise.
Substitutions are often written as $[\sigma_1\mapsto v_1,\dots,\sigma_n\mapsto v_n]$ which is to be understood as the partial function which maps $\sigma_i$ to $v_i$.

Let us define
$$ \State = \set{\Sigma_P\longto\Sigma^*}^+ $$
the set of all finite non-empty sequences of maps.

Let $ s= s_1\cdots s_n\in\State$ be a state, then define
\blist
    \item for $\sigma\in\Sigma_P$ we define $ s(\sigma)= s_n(\sigma)$,
    \item define $\pop s= s_1\cdots s_{n-1}$ (if $n>1$),
    \item define $\push s= s_1\cdots s_n s_n$,
    \item if $\nu$ is a substitution, define $s\nu$ to be $s_1\cdots s_{n-1}(s_n\nu)$.
\elist
\noindent So if we'd like to revert to a previous state, we simply pop from the current state.
And substituting the current state only alters the current (topmost) point state.

Now we begin with an initial $\beta$ function which is a partial function
$$ \beta\colon\overline\Sigma_A\times\overline\Sigma\times\State \plongto \overline\Sigma^*\times\State $$
Recall that $\overline\Sigma$ is $\Sigma\times{\bb N}$.
We will denote tuples in $X\times\State$ by $\gen{x,s}$ for $x\in X$ and $s\in\State$ for the sake of readability.
So we now wish to extend to a $\beta$ function
$$ \beta\colon\overline\Sigma^*\times\State \plongto \overline\Sigma^*\times\State $$
We do this as follows: given $\xi\in(\Sigma\times\overline{\bb N})^*$ and $s\in\State$ we define $\beta\gen{\xi,s}$ as follows:
\benum
    \item if $\xi=\sigma_i\xi'$ for $\sigma\in\Sigma_P$ then $\beta\gen{\xi,s}=\gen{s(\sigma)_i\xi',s}$,
    \item if $\xi=\sigma^1_i\sigma^2_j\xi'$ for $\sigma^1\in\Sigma_A$, $i\geq j$, such that $\beta\gen{\sigma^1_i\sigma^2_j,s}=\gen{\xi'',s'}$ is defined, then
        $\beta\gen{\xi,s}=\gen{\xi''\xi',s'}$,
    \item otherwise for $\xi=\sigma^1_i\sigma^2_j\xi'$, if $\beta\gen{\sigma^2_j\xi',s}=\gen{\xi'',s'}$ then $\beta\gen{\xi,s}=\gen{\sigma^1_i\xi'',s'}$.
\eenum
We also define the {\it initial priority function} to be a map $\pi\colon\Sigma_P\longto\overline{\bb N}$ (this is not a partial function: every printable symbol must be given a priority).
This is once again canonically extended to a function $\pi\colon\Sigma_P^*\longto(\Sigma_P\times\overline{\bb N})^*$.
And an {\it initial state} $s_0$ which is a point state.
The quintuple $(\Sigma_P,\Sigma_A,\beta,\pi,s_0)$ is called an {\it reducer}.
The reduction of a string $\xi\in S$ is the process of iteratively applying $\beta$ to $\gen{\pi(\xi),s_0}$.

\medskip
{\bf Example:\/} let
$$ \eqalign{
    \Sigma_P &= {\bb N}\cup\set{+,\cdot,=,;}\cup\set{{\tt let}}\cup\set{x^i}[i\in{\bb N}],\cr
    \Sigma_A &= {\bb N}\cup\set{(n+),(n\cdot)}[n\in{\bb N}]\cup\set{{\tt let}}\cup\set{({\tt let}x^i),({\tt let}x^i=)}[i\in{\bb N}]
} $$
where the natural numbers in $\Sigma_A$ are not the same as the natural numbers in $\Sigma_P$ since they must be disjoint, same for {\tt let}.
But they both essentially represent the same thing: $s_0$ maps $n\mapsto n$ for $n\in{\bb N}$ (the left-hand $n$ is in $\Sigma_p$, the right-hand $n$ is in $\Sigma_A$) and ${\tt let}\mapsto{\tt let}$.
All other printable symbols are mapped to $\epsilon$.

And similar to the previous example we define $\pi(n)=\infty$, $\pi(+)=1$, and $\pi(\cdot)=2$.
We extend this to $\pi(;)=0$, $\pi(=)=0$, $\pi({\tt let})=\infty$, and $\pi(x^i)=\infty$.

Let us take the same transitions as the example in the previous section for $n,(n+),(n\cdot)$ (we have to add the condition that the state doesnt change).
We further add the transitions
\medskip
\centerline{
    \vtop{\ialign{\hfil$#$\hfil\tabskip=.5cm&\hfil$#$\hfil\cr
        \gen{\sigma^1_i\sigma^2_j,s} & \beta\gen{\sigma^1\sigma^2,s}\cr\noalign{\kern3pt\hrule\kern3pt}
        \gen{\sigma;,s} & \gen{\sigma_j,s}\cr
        \gen{\lett x^i,s} & \gen{(\lett x^i)_j,s}\cr
        \gen{(\lett x^i)=,s} & \gen{(\lett x^i=)_j,s}\cr
        \gen{(\lett x^i=)\sigma,s} & \gen{\epsilon,s[x^i\mapsto\sigma]}\cr
}}}

\noindent In the final transition, $n\in\Sigma_A$.
Then for example (we will be skipping trivial reductions):
$$ \eqalign{
    \lett x^1=1+2;\ \lett x^2=2;\ x^1\cdot x^2; &\longto
    {\tt let}_\infty x^1_\infty=_01_\infty+_12_\infty;_0\ {\tt let}_\infty x^2_\infty =_0 2_\infty;_0\ x^1_\infty\cdot_2 x^2_\infty;_0\cr
    s_0\quad&\longto ({\tt let}x^1=)_01_\infty+_12_\infty;_0\ {\tt let}_\infty x^2_\infty =_0 2_\infty;_0\ x^1_\infty\cdot_2 x^2_\infty;_0\cr
    s_0\quad&\longto ({\tt let}x^1=)_03_0\ {\tt let}_\infty x^2_\infty =_0 2_\infty;_0\ x^1_\infty\cdot_2 x^2_\infty;_0\cr
    s_0[x^1\mapsto3]\quad&\longto {\tt let}_\infty x^2_\infty =_0 2_\infty;_0\ x^1_\infty\cdot_2 x^2_\infty;_0\cr
    s_0[x^1\mapsto3,\,x^2\mapsto2]\quad&\longto x^1_\infty\cdot_2 x^2_\infty;_0\cr
    s_0[x^1\mapsto3,\,x^2\mapsto2]\quad&\longto 3_\infty\cdot_2 x^2_\infty;_0\cr
    s_0[x^1\mapsto3,\,x^2\mapsto2]\quad&\longto (3\cdot)_2 x^2_\infty;_0\cr
    s_0[x^1\mapsto3,\,x^2\mapsto2]\quad&\longto (3\cdot)_2 2_\infty;_0\cr
    s_0[x^1\mapsto3,\,x^2\mapsto2]\quad&\longto 6_0\cr
} $$
\qedd

\medskip

{\bf Example:\/} Why does the initial $\beta$ function map to strings $\overline\Sigma^*\times\State$ and not $\Sigma\times\State$?
This is in order to implement functions; let us analyze the following example:
$$ \eqalign{
    \Sigma_P &= {\bb N}\cup\set{+,\cdot,=,(,),\{,\},\comm,;,{\tt let},{\tt fun}}\cup\set{x^i}[i\in{\bb N}]\cr
    \Sigma_A &= \Sigma_P^*
} $$
We will denote strings in $\Sigma_A$ with an underline to distinguish them from elements of $\Sigma_P$.
This is an extension of the previous examples, so the initial $\beta$ function acts the same on these characters, all we must do is add what happens to ${\tt fun}$ and the other characters we added.

\medskip
\centerline{
    \vtop{\ialign{\hfil$#$\hfil\tabskip=.5cm&\hfil$#$\hfil\cr
        \gen{\sigma^1_i\sigma^2_j,s} & \beta\gen{\sigma^1\sigma^2,s}\cr\noalign{\kern3pt\hrule\kern3pt}
        \gen{\{_i\sigma_j,s} & \gen{\sigma_j,\push s}\cr
        \gen{\
}}}

\bye

